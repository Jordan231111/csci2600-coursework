\documentclass[11pt]{article}
\usepackage{geometry}
\geometry{letterpaper, margin=1in}
\usepackage[utf8]{inputenc}
\usepackage[T1]{fontenc}
\usepackage{amsmath}
\usepackage{amssymb}
\usepackage{hyperref}

\hypersetup{
    colorlinks=true,
    linkcolor=blue,
    filecolor=magenta,
    urlcolor=cyan,
}

\setlength{\parindent}{0pt}
\setlength{\parskip}{1ex}

\begin{document}

\section*{Homework 5: Changes to hw4}

I made several strategic changes to the hw4 code to optimize performance and memory usage for the large-scale professor paths assignment:

\subsection*{Optimization of Graph Implementation}

\begin{itemize}
    \item \textbf{Disabled \texttt{checkRep()} for Performance}: The most significant change was ensuring the \texttt{CHECK\_REP\_ENABLED} flag in \texttt{Graph.java} was set to \texttt{false}. This optimization eliminated the expensive representation invariant checks during performance-critical operations, substantially improving both memory usage and execution time when working with large datasets.
    
    \item \textbf{Retained Core Functionality}: The core Graph ADT implementation and API remained unchanged, as the existing functionality provided all the necessary features:
    \begin{itemize}
        \item Adding nodes (professors) with string identifiers
        \item Adding directed edges with string labels (courses)
        \item Retrieving children with their edge labels (needed for BFS)
        \item Support for multiple edges between the same pair of nodes (allowing professors to teach multiple shared courses)
    \end{itemize}
\end{itemize}

\subsection*{BFS Implementation Optimizations}

\begin{itemize}
    \item \textbf{Lexicographical Path Finding}: Improved the breadth-first search algorithm by separating professor node sorting from course sorting. This ensured that paths are truly lexicographically ordered while minimizing performance impact.
    
    \item \textbf{Memory-Efficient Collection Usage}: Used efficient data structures for tracking visited nodes, maintaining the queue, and storing paths during BFS traversal. These optimizations were critical for processing large networks within the 256MB memory constraint.
    
    \item \textbf{Efficient Path Construction}: Implemented an efficient path construction mechanism that builds paths incrementally during traversal rather than reconstructing them afterward, reducing both time and memory overhead.
\end{itemize}

\subsection*{Test Dataset Generation}

\begin{itemize}
    \item Created Python scripts (\texttt{generate\_medium.py} and \texttt{generate\_larger.py}) for generating test datasets of different sizes to properly validate performance and memory usage.
    
    \item Generated a large test network with 5,000 professors and 1,000 courses that creates approximately 1.5 million potential edges, providing a rigorous test for the implementation.
\end{itemize}

These changes enabled the ProfessorPaths implementation to efficiently process large networks within the required memory constraints while maintaining correct functionality and performance requirements.

\end{document} 